\PassOptionsToPackage{unicode=true}{hyperref} % options for packages loaded elsewhere
\PassOptionsToPackage{hyphens}{url}
%
\documentclass[]{article}
\usepackage{lmodern}
\usepackage{amssymb,amsmath}
\usepackage{ifxetex,ifluatex}
\usepackage{fixltx2e} % provides \textsubscript
\ifnum 0\ifxetex 1\fi\ifluatex 1\fi=0 % if pdftex
  \usepackage[T1]{fontenc}
  \usepackage[utf8]{inputenc}
  \usepackage{textcomp} % provides euro and other symbols
\else % if luatex or xelatex
  \usepackage{unicode-math}
  \defaultfontfeatures{Ligatures=TeX,Scale=MatchLowercase}
\fi
% use upquote if available, for straight quotes in verbatim environments
\IfFileExists{upquote.sty}{\usepackage{upquote}}{}
% use microtype if available
\IfFileExists{microtype.sty}{%
\usepackage[]{microtype}
\UseMicrotypeSet[protrusion]{basicmath} % disable protrusion for tt fonts
}{}
\IfFileExists{parskip.sty}{%
\usepackage{parskip}
}{% else
\setlength{\parindent}{0pt}
\setlength{\parskip}{6pt plus 2pt minus 1pt}
}
\usepackage{hyperref}
\hypersetup{
            pdfborder={0 0 0},
            breaklinks=true}
\urlstyle{same}  % don't use monospace font for urls
\setlength{\emergencystretch}{3em}  % prevent overfull lines
\providecommand{\tightlist}{%
  \setlength{\itemsep}{0pt}\setlength{\parskip}{0pt}}
\setcounter{secnumdepth}{0}
% Redefines (sub)paragraphs to behave more like sections
\ifx\paragraph\undefined\else
\let\oldparagraph\paragraph
\renewcommand{\paragraph}[1]{\oldparagraph{#1}\mbox{}}
\fi
\ifx\subparagraph\undefined\else
\let\oldsubparagraph\subparagraph
\renewcommand{\subparagraph}[1]{\oldsubparagraph{#1}\mbox{}}
\fi

% set default figure placement to htbp
\makeatletter
\def\fps@figure{htbp}
\makeatother


\date{}

\begin{document}

We thank the reviewers for their comments, which have helped us to
improve the paper. We detail how we have handled their specific comments
in the rest of this document.

Reviewer 1:

Thank you for the notes. We have fixed both issues that you noted.

Reviewer 2:

We added a simple OpenMP example to the introduction to help
provide a simple introduction to OpenMP syntax.

We attempted to capture major additions to OpenMP in Figure 1 by marking
where tasks, SIMD and device offload were added.

Additions are generated in a variety of ways and for a variety of
reasons. We added a discussion of the process at a high level to the
beginning of the Evolution section of the paper.

We tweaked the sentence on the subject of directives allowing the user
to specify things the compiler cannot analyze, but the general point
remains true. In a variety of cases, the user has information the
compiler does not have for any number of reasons. Alias analysis in C
and C++ is a common example; another example is the ability to vectorize
a function call to a function that is not defined in the current
translation unit. Either of these can be made possible, or proven safe,
by user annotation in cases for which compiler analysis would be at the
very least impractical if not impossible.

Section 3.2 has been reworded slightly to attempt to better make the
point, which is that while serial equivalence is something we want to
enable users to produce with OpenMP features, the goal is not to make
OpenMP constructs serially equivalent by default due to the relatively
high additional costs that guarantee can require.

We have chosen not to change the captions. The comment involves an issue
of style. Every style guide with which the authors are familiar states
that captions are titles. They should be short; they should not be
paragraphs. Detailed explanations should be contained in the body text,
which must explain the figures and not just reference them. The body
text should stand on its own. While many writers of technical documents
choose to ignore these style guidelines, we do not agree with that
choice.

On the topic of descriptive vs prescriptive abstractions, while we
appreciate that the reviewer agrees with our position, many do not, at
least currently. We state our position in a long-running and rather
heated debate between the approaches taken by the OpenMP offload model
versus the one taken by OpenACC, among others. The discussion is an
important aspect of the principles that will shape the continued
evolution of OpenMP.

We agree that a list of OpenMP compilers and the versions/features that
they support is helpful. However, it is not within the scope of the
paper and could only capture one moment in time. So, we have added a
reference to a web page that hosts a listing of compilers with their
support along with a note on it at a high level. The page is regularly
updated and provides a better resource to readers overall.

\end{document}
