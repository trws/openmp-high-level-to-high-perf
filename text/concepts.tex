\section{Concepts and Mechanics}
\label{sec:concepts}

OpenMP has expanded greatly in scope and complexity since its inception, but
many of its features build on a common set of core mechanics and basic concepts
that have changed relatively little over the past twenty years.  This section
describes two of the most important building blocks of OpenMP, outlining and
data environments.

\subsection{Outlining}
\label{sub:outlining}

\begin{figure}
\begin{minted}{c}
void foo() {
  int a;
#pragma omp parallel
  {
#pragma omp master
    a = omp_get_num_threads();
  }
}
\end{minted}
\caption{Function that Uses OpenMP\label{fig:outline-before}}
\end{figure}

\begin{figure}
\begin{minted}{c}
struct foo_parallel_0 {
  int * a;
};

void foo_parallel_(void * data_in) {
  struct foo_parallel_0 * data = data_in;
  data->a[0] = omp_get_num_threads();
}

void foo () {
  int a;
  struct foo_parallel_0 data = {&a};
  runtime_parallel(foo_parallel_, &data);
}
\end{minted}
\caption{After Outlining\label{fig:outline-after}}
\end{figure}

Compiler outlining is the opposite of inlining. The technique extracts a
function from the body of another function. While conceptually simple, 
outlining forms the basis of the most common implementation of most OpenMP 
constructs that transform serial code to run in parallel. Specifically it 
allows the compiler to create the functions required as targets for underlying
threading primitives. For example, an implementation may convert a parallel 
region like that in Figure~\ref{fig:outline-before} into a new function 
and runtime calls as in Figure~\ref{fig:outline-after}.

While our example is simplified, the transformation can outline any block 
into a function with an appropriate signature for specific parallelization 
mechanisms and capture any necessary state in a compatible data structure 
or type. Thus, the user does not need to create wrapper functions and 
single-use structures to encapsulate their code in order to parallelize it. 
Instead, the compiler does the repetitive work, while the user determines the
appropriate form and granularity of parallelism. This technique allows 
the user to specify something for which compiler analysis is highly complex 
while allowing the compiler to handle the repetitive and error-prone 
portion of the transformation, which is the most enduring aspect of the 
philosophy of OpenMP, as stated in Section~\ref{sec:philosophy}, 

Interestingly, C++11, with lambdas, and C, with the blocks extension, now 
provide outlining mechanisms directly to the user. Thus, these languages
can cover many original OpenMP features (many of the newer OpenMP features
would require additional extensions). We revisit this technique and its
relationship to the base languages in Section~\ref{sec:future_directions} when
we discuss some potential directions for the continuing evolution of OpenMP.

\subsection{Data Environments}
\label{sub:data_environments}

While outlining supports parallelization, it does not directly address 
the issue of data sharing between threads or tasks. In OpenMP, every task, 
including implicit tasks that a loop or device construct creates, has its 
own data environment that represents its view of memory and of state in the 
OpenMP runtime. The simplest manifestations of a data environment provide 
variables that are private to the task, thread, team or construct in general 
without having to refactor variable declarations and initializations in 
user code.

OpenMP data environments also include ICVs~(Internal Control Variables),
which are a less familiar but equally important aspect of them. ICVs govern
the actions of the OpenMP runtime. Most users know some of the major ICVs by 
their associated environment variables, such as \texttt{OMP\_NUM\_THREADS}, 
and the behavior of environment variables and data environments are similar. 
Each new data environment inherits some values and their behaviors from the 
enclosing data environment but is otherwise independent of that enclosing 
environment. Thus, each task can control the behavior of OpenMP in its 
dynamic scope without changing the behavior of OpenMP constructs outside 
of that scope. Thus, the mechanism supports composability and control.

Overall, OpenMP data environments are an essential concept that has evolved
with OpenMP. For example, we added the concept of separate data environments 
for each device along with the device constructs. This concept provides a 
richer memory environment than the original OpenMP shared memory environment. 
Specifically, distinct device data environments can have copies of the same 
variable that may share storage -- or may not. Thus, OpenMP provides 
mechanisms to keep the potential copies consistent.
