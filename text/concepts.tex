\section{Concepts and Mechanics}
\label{sec:concepts}

OpenMP has expanded greatly in scope and complexity since its inception, but
many of the features build on a common set of core mechanics and basic concepts
that have changed relatively little over the past twenty years.  This section
describes two of the most important building blocks of OpenMP, outlining and
data environments.

\subsection{Outlining}
\label{sub:outlining}

\begin{figure}
\begin{minted}{c}
void example() {
  int a;
  #pragma omp parallel
  {
    #pragma omp master
    a = omp_get_num_threads();
  }
}
\end{minted}
\caption{A function that uses OpenMP.\label{fig:outline-before}}
\end{figure}

\begin{figure}
\begin{minted}{c}
struct foo_parallel_0 {
  int * a;
};

void foo_parallel__(void * data_in) {
  struct foo_parallel_0 * data = data_in;
  data->a[0] = omp_get_num_threads();
}

void foo () {
  int a;
  struct foo_parallel_0 data = {&a};
  runtime_parallel(foo_parallel__, &data);
}
\end{minted}
\caption{After outlining is applied.\label{fig:outline-after}}
\end{figure}

Compiler outlining is the opposite of inlining. The technique extracts a 
function from the body of another function. The technique forms the basis 
of the most common implementation of most OpenMP constructs, which transform
serial code to run in parallel. By automatically creating the 
functions required as targets for the underlying operating system (OS) 
threading primitives. For example, an implementation might convert a 
parallel region like the one in Figure~\ref{fig:outline-before} into a 
new function and a call, or calls, into the runtime as in 
Figure~\ref{fig:outline-after}.

While our example is simplified, the general transformation to outline any 
block generates a structure that holds all necessary state and a function 
with a signature compatible with the threading abstraction in use. Thus, 
the user need not manually create wrapper functions and single-use structures 
to encapsulate their code whenever something should be run in parallel or 
made into a task. Instead, the compiler does the repetitive work and leaves
the task of locating appropriate granularities of parallelism to the user.
This technique supports allowing the user to specify something for which the
compiler analysis is highly complex while allowing the compiler to handle the
repetitive and error-prone portion of the transformation, which is the most 
enduring aspect of the philosophy of OpenMP, as stated in 
Section~\ref{sec:philosophy}, 

Interestingly, C++11, with lambdas, and C, with the blocks extension, now 
provide outlining mechanisms directly to the user. Thus, these languages
can cover many original OpenMP features (many of the newer OpenMP features
would require additional extensions). We revisit this technique and its
relationship to the base languages in Section~\ref{sec:future_directions} 
when we discuss some of the continuing evolution of OpenMP.

\subsection{Data Environments}
\label{sub:data_environments}

While outlining supports parallelization of code written in a serial style, 
it does not directly address the issue of data sharing between threads or 
tasks. In OpenMP, every task, including implicit tasks that a loop or 
device construct creates, has its own data environment that represents its 
view of memory and of state in the OpenMP runtime. The simplest manifestations
of the data environment provide variables that are private to the task, 
thread, team or construct in general without having to re-factor the 
declaration and initialization of the variables in user code.

OpenMP data environments also include ICVs~(Internal Control Variables),
which are a less familiar but equally important aspect of them. ICVs govern
the actions of the OpenMP runtime. Most users know some of the major ICVs by 
their associated environment variables, such as \texttt{OMP\_NUM\_THREADS}, 
and the behavior of environment variables and data environments are similar. 
Each new data environment inherits the values or behaviors of the enclosing 
data environment, but it is otherwise independent of the enclosing environment.
Thus, each task can control the behavior of OpenMP in its dynamic scope 
without changing the behavior of OpenMP constructs outside of that scope.
Thus, the mechanism supports composability and control.

Overall, OpenMP data environments are an essential concept. As OpenMP
has evolved, the data environment concept has also. For example, with 
the addition of device constructs, we added the concept of separate 
data environments for each device. This concept provides richer memory
environment than the original OpenMP shared memory environment.Specifically,
distinct device data environments can have copies of the same variable
that may share storage -- or may not. Thus, the use must use OpenMP 
constructs to ensure that the potential copies are consistent.
