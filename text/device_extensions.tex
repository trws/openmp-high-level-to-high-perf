\subsection{Device Extensions}
\label{sub:device_extensions}

While OpenMP introduced support to offload computation regions to target 
devices in version 4.0 and subsequently expanded that support significantly 
in 4.5, the space is changing quickly. Thus, we have already adopted several
extensions and refinements for OpenMP 5.0 including changes that greatly 
simplify the use of functions in those regions. Further, a new general 
mechanism to specify application-specific requirements will enable 
straightforward use of unified memory spaces across devices, Nonetheless, 
we have also adopted a unique deep-copy mechanism that will significantly
improve usability on systems that do not provide unified memory spaces.
Importantly, we expect this deep-copy support will often provide performance 
advantages even on systems that do provide them.

Many offload models, such as CUDA and OpenCL, require function annotations. 
However, OpenMP 5.0 will ease the use of functions on devices by relaxing 
its annotation requirements. OpenMP~5.0 will eliminate the requirement to 
annotate function declarations. Essentially, the compiler must assume that 
a device variant will be available at link time. Also, the compiler must 
automatically generate a device variant for any function with a definition 
in the same translation unit as a call from a \texttt{target} region. 
Essentially, the definition implicitly includes the \texttt{declare target} 
annotation. Because these changes significantly improve usability, many 
compilers have already implemented them and they have allowed entire large 
codebases (particularly in C++ due to the pervasiveness of templates) to 
offload to devices using OpenMP without a single explicit \texttt{declare 
target} directive; other models require hundreds or thousands of annotations 
to compile them at all.

In order to assume coherent memory between the host and a target device, 
the user must assert to the compiler that their code requires that support.
Given this assertion, if the code is run on a device without that support, 
it may exhibit unspecified behavior (i.e., the code is broken). Overall,
these assertions are a contract between the application and the compiler, 
which is a general mechanism for which unified memory spaces are just one 
instance. Thus, OpenMP~5.0 will provide a new \texttt{requires} directive 
that allows OpenMP to specify a set of rules for a given requirement and 
users to specify that their code conforms to those rules. This directive 
supports the definition of subsets of the OpenMP specification; one 5.0
subset will support systems that do not require memory to be mapped explicitly
into a data environment for target devices. Effectively, the user can assume 
shared memory between the host and the devices. For example, the code in 
Figure~\ref{fig:unified} is only valid for systems with a unified view of 
memory. It is non-conforming in OpenMP up to 4.5 but will be correct on 
systems that meet the requirement. Importantly, the \texttt{requires} 
directive applies to an entire translation unit, which offers usability 
benefits similar to the implicit \texttt{declare target} annotations.

\begin{figure}
\begin{minted}{c}
#pragma omp requires        \
        unified_shared_memory
struct list {
  void * data;
  struct list * next;
};

void foo() {
  struct list * l = make_linked_list();
#pragma omp target
  {
    struct list * cur;
    while(cur) {
      do_something_with_data(cur->data);
      cur = cur->next;
    }
  }
}
\end{minted}
\caption{Requiring Unified Memory\label{fig:unified}}
\end{figure}

The deep-copy support in OpenMP 5.0 will simplify the use of pointer-based 
data structures like the linked list in Figure~\ref{fig:unified} on systems 
that do not provide coherent unified memory. With OpenMP~4.5, the user must 
map each piece of the structure and must then assign the pointers on the 
device to those pieces either with explicit assignments or with further 
mapping actions. The user often must repeat this verbose, complex and error 
prone code sequence every time an instance of the data structure is needed 
on the device. Instead, the \texttt{declare mapper} directive in OpenMP 5.0 
will allow the user to describe how to map an instance of the data structure 
including the targets of pointers. The user can then use this definition in 
a \texttt{map} clause whenever an instance of the data structure is needed 
on the device. Overall, the descriptions in the \texttt{declare mapper} 
directive are simpler than the OpenMP~4.5 mechanism and eliminate the 
repetition. Figure~\ref{fig:mapper} shows an example that maps a multi-level 
data structure with the \texttt{declare mapper} directive.  The directive in 
the \texttt{Vec} class uses a \texttt{map} clause to describe how to map the 
data that is the target of the pointer member for any instance of the class.  
This version works for any target platform, including those that do not 
support unified memory.

\begin{figure}
\begin{minted}{cpp}
class Vec {
  size_t len;
  double * data;
public:
  // Normal vector methods
  #pragma omp declare mapper(Vec v) \
    use_by_default                  \
    map(v, v.data[0:v.len])
};

void foo() {
  Vec v1(100), v2(100);
  fill_vec(v1);
  #pragma omp target teams distribute \
      parallel for                    \
      map(to:v1)   // explicitly map v1
  for(auto i = 0; i < v1.size(); ++i) { 
    v2[i] = v1[i]; // implicitly map v2
  }
  // v2[0-100] == v1[0-100]
}
\end{minted}
\caption{User-Defined Mapper Example\label{fig:mapper}}
\end{figure}

We plan to refine the deep-copy mechanism further. Specifically, we will 
provide a mechanism that can replace any phase of the mapping process with 
user-defined expressions or functions written in the base language. This 
mechanism, which will provide equivalent functionality to data  serialization 
and deserialization for transmission over a network,  will support mapping 
of arbitrary, complex data structures. Further, it will enable data-dependent 
data transformations that support highly efficient kernel computations. We
are working to include this functionality in OpenMP 5.0 although we may
have to defer its specification for OpenMP 5.1.




