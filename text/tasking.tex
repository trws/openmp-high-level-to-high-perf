\subsection{Tasking Extensions}
\label{sub:tasking}

%% BRONIS: We should  cover 4.0 and 4.5 tasking extensions
%% BRONIS: Specifically: Discuss task dependences, taskgroup and taskloop

% the subsubsection is just for my orientation and can be removed.
\subsubsection{Taskloop}
\label{sec:Taskloop}

Support for task-based loops, or \emph{taskloops} for short, is  an extension that was made in OpenMP 4.5 to ease using the tasking model with loop-based parallelism.
While it was always possible to manually cut a loop into chunks and assign them to OpenMP tasks, it was a cumbersome and error-prone code transformation process that had to be applied by the programmer.
The OpenMP~4.0 \code{taskloop} construct provides syntactic sugar to automatically transform a loop into a parallel loop using OpenMP task.

\begin{figure}
\begin{minted}{c}
void sapxy_tasks(float* a, float* b,
                 float s, size_t n)
{
  size_t i;
#pragma omp taskloop simd \
            num_tasks(NTASKS) \
            shared(a,b) firstprivate(s)
  for (i = 0; i < n; i++) {
    a[i] = a[i] * b[i] * s;
  }
}
\end{minted}
\caption{Example for using the \code{taskloop} construct.\label{fig:TaskloopExample}}
\end{figure}

Figure~\ref{fig:TaskloopExample} shows how to apply the \code{taskloop} construct to parallelize a \emph{saxpy} operation.
The \code{num\_tasks} clause is used to specify how many OpenMP tasks shall be created for the sapxy loop.
Alternatively, programmers can use the \code{grainsize} clause to inform the OpenMP implementation about the minimal number of loop iterations per task.
The \code{taskloop} construct is also available as a combined construct to use SIMD instructions for the tasks resulting from the loop.
