\subsection{Concurrency and Event-Based Model}
\label{sub:concurrency_and_event_based_model}

%% BRONIS: PLACEHOLDER

OpenMP initially focused primarily on deploying concurrency for 
performance using parallel execution. It has evolved to support 
the management of concurrency for broader use, as evidenced by 
the addition of task parallelism for asynchronous execution. 
One form of concurrency not yet addressed is the event-driven 
model used in many interactive applications.  In this model, a  
thread runs continuously in an event loop to process user 
actions, generating work in response to those actions.  
 
The main obstacle to implementation of the event-based model in 
OpenMP is the specification's requirement that the thread 
encountering a parallel construct become the master thread of 
the new team generated by the resulting parallel region.  To 
address this obstacle, a new capability could be introduced to 
allow a thread to direct work toward a team other than its own. 
Such a capbility would allow the event thread to remain 
responsive as other teams concurrently handle processing of 
previous events.

