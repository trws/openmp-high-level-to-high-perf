\subsection{Concurrency and Event-Based Model}
\label{sub:concurrency_and_event_based_model}

%% BRONIS: PLACEHOLDER

OpenMP initially focused primarily on deploying concurrency for performance
using parallel execution. It has evolved to support the management of
concurrency for broader use, as evidenced by the addition of task parallelism
for asynchronous execution.  One form of concurrency not yet addressed is the
event-driven model used in many interactive applications and networking servers.
In this model, one or more threads run continuously in an event loop to process
user actions, generating work in response to those actions.  
 
The main obstacles to implementation of the event-based model in 
OpenMP is the specification's requirement that the thread 
encountering a parallel construct become the master thread of 
the new team generated by the resulting parallel region and the generation of a
new data environment and task context for each task.  To 
address this obstacle, a new capability could be introduced to 
allow a thread to direct work toward a team other than its own. 
Such a capability would allow the event thread to remain 
responsive as other teams concurrently handle processing of 
previous events.  Free-agent threads solve some of this problem, but not
entirely, since they still incur the overhead of creating a new task on each
creation.  We intend to explore methods to create re-usable tasks among other
methods to mitigate this expense.

