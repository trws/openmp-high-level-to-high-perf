\section{Implementation}
\label{sec:imp}

We implement \tsar as a C library on top of Accelerated OpenMP, tested with \pgia
and Cray's Accelerated OpenMP. Our evaluation in this paper
focuses on \pgia, so our examples use its directive format.  This section
discusses our implementation including its portability, API and our memory
manager as well as some necessary deviations from the design discussed in
Section~\ref{sec:design}.

\subsection{GPU Back-off}
\label{sec:averse}

Some applications are not amenable to being run on GPUs, or at least the GPUs
present in some systems. While iterations of a region may benefit greatly from
running on an NVIDIA c2075, they may perform poorly on an NVIDIA GeForce GT
520. In order to maintain portability across disparate accelerator and CPU
capabilities, \tsar implements GPU Back-off support in all adapting
schedulers.

The back-off system is implemented differently for each of the two scheduler
types.  In the adaptive schedulers \tsar converts a GPU offload thread into a
CPU thread when the GPU has a higher time per iteration than the slowest CPU
core for a configurable number of passes (default is two). We use multiple
iterations since under certain circumstances, such as loading large persistent
datasets for the first time, or an inappropriate initial amount of work, a
device can be erroneously classified as slower than the CPUs.  With the chunk
schedulers, we base the decision on whether a given GPU completes fewer
iterations than the slowest CPU core during each pass of a configurable number
of passes. This difference compensates for the sometimes highly variable time
per iteration when bootstrapping chunk schedulers across initial data copies,
which can cause false conversions with the adaptive back-off scheme.  We
discuss the effects of this extension further in Section~\ref{sec:results}.

\subsection{Memory Management}
\label{sec:memory}

The existing memory interface of Accelerated OpenMP is insufficient to express
the relationships necessary to handle certain kinds of memory association.
While Accelerated OpenMP does natively support copying a subset of an array,
it does not support copying multiple subsets of one array, nor does it support
non-contiguous rectangular sections such as a subset of the columns of a 2D
array.

In order to support our desired memory association interface, \tsar
implements its own memory manager, using the \verb#deviceptr()# clause to pass
\tsar managed memory into Accelerated OpenMP regions.  We offer a
straightforward syntax by which users specify the data required by a given
iteration.  Given that information, \tsar automatically copies the ranges of
data required by whatever iterations are assigned to a given device for that
pass.  When possible the memory manager uses pinned memory to accelerate
copies, as well as asynchronous copies to and from the device in order to
overlap them with scheduling and synchronization.

Currently, the \tsar memory manager handles a restricted set of partial
copies.  In addition to the straightforward one-to-one relations, \tsar also
supports stencils through padding, and row, column or planar associations on
two and three dimensional matrices.  In order to support reductions we provide
an API inspired by user-defined reductions in OpenMP 4.0. We discuss the
details further in Section~\ref{sec:api}. While only a subset of the possible
cases, these mappings are sufficient to implement all benchmarks evaluated in
this paper.

\subsection{Data Packing and Padding}
\label{sec:packing}

%TODO: add discussion of observed iterator values
%TODO: add discussion of column-wise packing for known-shape accesses vs.
%      dynamic multi-dimensional arrays

% query ratio off by 0: 1.08
% query ratio off by 1: 1.848359687
% query ratio off by 8: 1.6

Our original implementation of the memory interface~\cite{scogland:7Hpt64iV} 
had a material weakness. That version of \tsar
allocated the full size of each memory region on each device in order to
preserve offset accuracy. In other words, any input or output array/matrix
supplied to \tsar was allocated in full in all participating address spaces.
Managing subset allocation and access without invalidating offsets and
iterator values is a difficult problem, especially in languages like C. 

We have redesigned \tsar to support three kinds of regions depending on how
the data is mapped.  The first, and most simple, case is a one dimensional
partial array or two dimensional array that is associated by rows.  Since all
of the resulting subsets are contiguous, the runtime provides an offset
pointer that can be indexed by the original offsets without issue.  No further
action or overhead is required for this case, and a significant amount of
storage on accelerators can be saved.  The second case is where a two or more
-dimensional array is associated by columns.  \tsar can pack these, but must
have control over the calculation of offsets into the resulting matrix.  As
such, we handle this case in our translator for contiguous arrays accessed
with \verb#array[i][j]# style syntax, but currently do not support 
dynamically-allocated C arrays accessed with the \verb#array[i*row_size + j]# 
syntax, though these can be supported by directly using the C API functions.  
Third, associations can use both the row and column associations, resulting 
in a region resembling a plus-sign being assigned to each device.  Since these
require the full range in both rows and columns, even though they may not need
the corners, \tsar is forced to allocate the full size of such arrays.

By allowing data regions to be packed, \tsar gains two extra capabilities
beyond reducing memory usage on target devices.  The packing functionality
allows any chunk-based scheme to place a low bound on memory use by
selecting a small chunk size. This allows large data sets on the host to
be streamed through accelerators without enough memory to hold even their
assigned sub-part of the problem.  When used in this mode however, \tsar
becomes similar to a blocked-task system, including the increased task
management and data-transfer overhead that implies.

Perhaps more importantly however, the capability to adjust indexing, as
described for column-wise associated multi-dimensional arrays, allows regions
not only to be shrunk to save space, but also padded for alignment. As is
well known, memory alignment is important for the performance of
SIMD computations and coalesced memory accesses are important
to the performance of GPU kernels.  Given the ability to pad rows beyond the
data assigned to each device, or even rows of data that are mis-aligned by the
user, \tsar can ensure that each row is aligned for most efficient access on
each target. We implement this optimization by ensuring that the length of
each row in a matrix is a multiple of the target device's SIMD width.  While
in some cases this choice is more strict than required, it is consistently
sufficient to ensure reasonable alignment.

Figure~\ref{fig:reshape} shows the effect that even small changes in row-width
can have on performance without padding, and how our auto-padding helps.  The
figure represents the performance of a general matrix multiplication kernel
when run with row and column lengths ranging from 8,192 to 8,223 elements in
increments of one, specifically the x-axis values are the number of elements
over 8,192 in each row.  On our primary evaluation system, with four c2070
GPUs, a square matrix of size $8192\times8192$ runs in 7.9 seconds.
Increasing that size by only one element on each side more than doubles that
runtime to 19.5 seconds.  In fact, every odd-numbered increase in size takes
approximately the same increased time, while each power of two increase does
better up to 16, or a total of 8208, which performs the same as the original
8192.  Another system with a pair of k20x GPUs shows a nearly identical
stepped pattern as well.  The difference in performance is somewhat smaller,
ranging from 10.9 seconds at the zero and 16 positions and 15.2 at the odd
offsets.The L2 read request performance counters provide a partial explanation
for the wide range in performance.  When padding is enabled, the difference in
L2 requests with a multiple of 16 row length is consistently within 10\%; for
any odd length it balloons to 80\% more L2 accesses for the un-padded version.
This increase is due to a greater number of reads being required to
accommodate the mis-aligned read requests of each warp on the GPUs, increasing
contention and lowering cache efficiency overall.  On the right-hand-side of
the plot, you can see that \tsar's automatic padding smooths out these issues.
Also, when dividing a data-set column-wise, this padding support can ensure
alignment even when the appropriate amount of work to be assigned is not a
multiple of the target device's native SIMD width, an important consideration
for several of our benchmarks.

\begin{figure}[t]
        \includegraphics[width=\columnwidth]{gemm_reshape_range}
        \caption{Runtime of GEMM kernel on square matrices, statically
          scheduled across GPUs only with and without auto-padding}
        \label{fig:reshape}
\end{figure}


\subsection{\tsar API and Usage}
\label{sec:using}
\label{sec:api}

% \begin{figure}[t]
%     \begin{center}
%         \ifm
%         \inputminted[fontsize=\scriptsize,frame=single]{c}{snippets/api.c}
%         \else
%         \lstinputlisting{snippets/api.c}
%         \fi
% 	\end{center}
%     \caption{\tsar API}
%     \label{fig:api}
% \end{figure}

This section describes the low-level API to the \tsar library in detail.

\textbf{ctsar\_init} Initializes an instance of the \tsar runtime, one such
instance should be used for each region that is to be separately scheduled.
The parameters allow a user to set the default scheduler, allowed devices, the
default time per iteration for each accelerator as an array of doubles (NULL
for defaults), and how finely the split and quick schedulers should divide
regions (NULL for default).

\textbf{ctsar\_next} Computes the division of work for the region associated
with \verb#c# based on \verb#size# total iterations.  This function is also
responsible, updating appropriate memory regions on each target device and
starting timers to evaluate each device's performance.

\textbf{ctsar\_loop} In order to support \emph{split}, \emph{quick} and the
chunk schedulers, \tsar must reevaluate the loop with each thread repeatedly.
The \verb#ctsar_loop# function serves as the condition for a do/while loop
surrounding each region. In addition to managing repeats, the loop function is
responsible for synchronization, GPU back-off support, copying data back from
all devices, completing reductions, and calculating performance statistics at
the end of each pass.

\textbf{ctsar\_reg\_mem\{\_2d\}} These functions register a host buffer with
\tsar.  The full version takes a pointer to CPU memory, the size of an element
of the input data, the number of element in each row/column, the number of
halo elements required, and a flag option that allows the user to control
copy direction and type. The non-2D version is shorthand for 1D arrays.  The
return value is a pointer to the memory assigned to the calling thread, which
may or may not be identical to the original pointer.

The flags value controls whether memory is copied in or out or both, as well
as whether to copy persistently, partially by rows or partially by columns and
whether padding is to be allowed, if it is, an extra output parameter for the
new row size is required. Partial copies are integral to the correct
functioning of \tsar as they make automatic merging of output possible. They
also improve performance of input operations. The 2D interface supports all
specifications discussed in Section~\ref{sec:design}, except that it does not
handle matrices with dimensionality higher than 2.

Regardless of the flags, \tsar allocates an appropriate size buffer on the
device associated with the calling thread. If the region is set to persistent,
data is immediately and asynchronously copied from the CPU array into the newly
allocated memory, where it resides until it is explicitly removed with a call to
\verb#ctsar_unreg_mem()#.

\textbf{ctsar\_unreg\_mem} De-registers the pointer from the region instance,
frees the memory that stores the state of the data, and frees persistent
regions.

\textbf{ctsar\_retarget\_mem} Re-target allows a user to specify that the
region already allocated for pointer \verb#old# should be used to store the
data pointed to by \verb#new# on all devices.  A typical use is to swap
buffers for double buffering, although it can also be used to implement
blocked data transfers by re-targeting a pointer to the new start pointer
before entering a region.

\textbf{ctsar\_reg\_reduc} This function registers a reduction. Because each
memory space will have its own reduction result, \tsar must safely initialize
the temporary variables in each memory space and combine those results into a
meaningful final value. The identity pointer points to an appropriate initial
value to use on each device. For example, in a sum the identity would usually
be 0, in a product 1, and so on.  The \verb#item_size# specifies the size of
the elements to be reduced.  The \verb#reduc# argument is function pointer
that should accept two void pointer arguments, the first of which is both a
value to be reduced and the output, the second is another value to be reduced.
This function is called repeatedly to accumulate the final value as each
device finishes execution.  For simple reductions, the body of the function
can be as simple as \verb#*((int*)a)+=*((int*)b)#.

\textbf{CSTART/CEND} Macros used to retrieve the start and end values to use
for iteration in the loop region.




%\subsubsection{Example}

\begin{figure}[t]
    \begin{center}
        \ifm
        \inputminted[fontsize=\scriptsize,frame=single]{c}{snippets/manual.c}
        \else
        \lstinputlisting{snippets/manual.c}
        \fi

        % \ifm
        % \inputminted[fontsize=\scriptsize,frame=single]{c}{snippets/auto.c}
        % \else
        % \lstinputlisting{snippets/auto.c}
        % \fi
    \end{center}
    \caption{CoreTSAR library version of GEMM}
    \label{fig:k-means-split}
\end{figure}

Figure~\ref{fig:k-means-split} presents an example using this interface to
implement the extension as presented in Figure~\ref{fig:ctsar-ext}. In this
example, \tsar is initialized with the adaptive scheduler, default ratio, and
div of 10. The parallel do-while loop allows our library to reevaluate the
code region as necessary by looping with the \verb#ctsar_loop(s)# call until
done. The data regions are registered, as partial input, complete input, and
partial input/output, and the appropriate pointers for those data regions on
each device are returned into the local copies of pointers \verb#a#, \verb#b#
and \verb#c#. The \verb#ctsar_next()# call calculates the number of iterations
to be completed in this pass by each device. Once it is complete, the
\verb#CSTART()# and \verb#CEND()# macros return the appropriate iterator range
values for the device that evaluates them.  This syntax can either be used
manually, or generated by our python/libclang-based source-to-source
translator.

While the code is extended significantly around the loop, we do not replicate
or alter any code in the loop body. The Accelerated OpenMP \verb#if()# clause
determines if a thread runs on a GPU or CPU core. If the device is a CPU, the
loop is run serially on the associated core completing its assigned
iterations. If it is a GPU-controlling thread, the \verb#acc region# directive
workshares the assigned iterations across the associated GPU.  All codes used
in our evaluation are implemented in this fashion.

% \subsection{Portability}
% \label{sec:port}

% Currently \tsar assumes that OpenMP manages CPU threads, and that accelerators
% are CUDA-enabled GPUs. Thus, we use \verb#omp_get_thread_limit()# to
% determine the number of CPU resources, \verb#cudaMalloc()/cudaMemcopy()# to
% allocate and to manage accelerator memory and \verb#cudaGetDeviceProperties()#
% to read the number of cores and capabilities of the GPU for initial performance
% estimation. Our design does not require these assumptions so we could easily
% add other targets or replace these interfaces.


