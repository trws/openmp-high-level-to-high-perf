\subsection{Serial Equivalence}
\label{sub:serial_equivalence}

An original goal for OpenMP was to support serial equivalence to the 
greatest possible extent. As a result, many think that all OpenMP programs, 
or at least all correct OpenMP programs, are guaranteed the same result
if the code is executed in parallel as when the compiler completely 
ignores all OpenMP constructs. However, even OpenMP 1.0 included runtime
functions that allow a program to depend on the number of threads or the
thread number (or ID) that executed a region. Thus, trivial programs could
fail to exhibit serial equivalence. Today, many more opportunities exist
to write OpenMP programs that do not provide serial equivalence. 

As OpenMP has evolved, the opportunities to write programs that do not
exhibit serial equivalence have increased. Figure~\ref{fig:trivial_task} 
provides a simple tasking program in which the serial version has an infinite 
loop while the parallel version will complete quickly, assuming that the 
parallel region uses two or more threads and different threads execute the 
two tasks. Figure~\ref{fig:trivial_target} shows a simple example for 
accelerators in which ``incremented'' is always printed, while 
``incremented again'' may or may not print with OpenMP, depending on 
whether the host and accelerator share memory. 

\begin{figure}
\begin{minted}{c}
void example() {
  int a = 0;
#pragma omp parallel
  {
#pragma omp single
    {
      int b = 0;
#pragma omp task
      while (b == 0) {
#pragma omp atomic read seq_cst
        b = a;
      }
#pragma omp task
      {
#pragma omp atomic update seq_cst
        a++;
      }
    }
  }
}
\end{minted}
\caption{Trivial OpenMP Program\label{fig:trivial_task}}
\end{figure}

\begin{figure}
\begin{minted}{c}
void example2() {
 int a = 0;
#pragma omp target map(tofrom:a)
  {
    a++;
  }

  if (a)
    printf("incremented\n");
  
#pragma omp target map(to:a)
  {
    a++;
  }

  if (a == 2)
    printf("incremented again\n");
}
\end{minted}
\caption{Trivial OpenMP Accelerator Program\label{fig:trivial_target}}
\end{figure}


In general, serial equivalence requires the program or runtime to limit
the possible execution orders. As OpenMP has grown to support more parallel 
programming patterns, the range of execution orders has also grown, which 
implies more opportunities not to exhibit serial equivalence or would require
more execution order limitations, which would limit performance. OpenMP tries 
to avoid those limitations unless the programmer requires them. Thus, the 
philosophy of OpenMP remains to provide constructs that simplify the 
enforcement of serial equivalence when desired but has evolved not 
to limit parallelism and, by extension, performance unnecessarily.

