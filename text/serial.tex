\subsection{Serial Equivalence}
\label{sub:serial_equivalence}

A original goal for OpenMP was to support serial equivalence to the 
greatest possible extent. As a result, many think that all OpenMP programs, 
or at least all correct OpenMP programs, are guaranteed the same result
if the code is executed in parallel as when the compiler completely 
ignores all OpenMP constructs. However, even OpenMP 1.0 included runtime
functions that allow a program to depend on the number of threads or the
thread number (or ID) that executed a region. Thus, trivial programs could
fail to exhibit serial equivalence. Today, many more opportunities exist
to write OpenMP programs that do not provide serial equivalence. 
%% BRONIS:Could include the trivial example here

%% BRONIS: Tom, you mentioned data privatization in the text that I
%% BRONIS: deleted. Do you have an interesting example in mind?

As OpenMP has evolved, the opportunities to write programs that do not
exhibit serial equivalence have increased. OpenMP support for tasking 
provides numerous opportunities. Figure~\ref{fig:trivial_task} provides
a simple tasking program in which the serial version has an infinite loop
while the parallel version will complete quickly, assuming that the parallel
region uses two or more threads and different threads execute the two tasks.

\begin{figure}
\begin{minted}{c}
int a = 0;
#pragma omp parallel
{
  #pragma omp single 
  {
    int b = 0;
    #pragma task
    while (b == 0) {
      #pragma atomic read seq_cst
      b = a;
    }
    #pragma task
    {
      #pragma atomic update seq_cst
      a++;
    }
  }
}
\end{minted}
\caption{Trivial Tasking Program}
\label{fig:trivial_task}
\end{figure}

%% BRONIS: Can we provide another exmple, say that uses device constructs?

In general, serial equivalence requires the program or runtime to limit
the possible execution orders. As OpenMP has grown to support more parallel 
programming patterns, the range of execution orders has as well, which 
implies either more opportunities to escheew serial equivalence or more
limitations on the range of execution orders. Since such limitations would 
also create performance limitations, OpenMP (or any parallel programming 
system) tries to avoid them unless the programmer requires them. Thus, the 
philosophy of OpenMP remains to provide constructs that simplify the 
enforcement of serial equivalence when desired but that has evolved not 
to limit parallelism -- and by extension -- performance unnecessarily.

