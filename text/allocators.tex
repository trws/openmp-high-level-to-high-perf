\subsection{Allocators and Hierarchical Memory}
\label{sub:allocators_and_hierarchical_memory}

Memory hierarchies are expected to become deeper in future systems with the
use of technologies such as high-bandwidth memory and non-volatile RAM. Each 
of these technologies has a different programming interface and distinct
performance characteristics. Programming mechanisms must address these 
differences and support intelligent data placement since the fastest resources
typically have limited capacity. To enable programmability of these 
technologies and portability across platforms, OpenMP~5.0 will include a 
consistent and portable interface for placement within the memory hierarchy.

The term \emph{memory space} refers to a memory resource available in the
system when the OpenMP program is executed. Memory spaces differ in their 
characteristics, for instance in bandwidth or capacity. An \emph{allocator} 
is an object that allocates and frees memory from a suitable memory space 
and OpenMP will provide intuitive pre-defined memory allocators. The use of 
an allocator expresses the intent to allocate memory from a memory space 
with certain properties. For example, the pre-defined memory allocators 
can select a memory space with large capacity, high bandwidth or low latency, 
or memory local to a particular thread or thread team.

OpenMP 5.0 will include the \texttt{omp\_alloc} and \texttt{omp\_free} 
routines as direct replacements for \texttt{malloc} and \texttt{free} from 
the standard library. The \texttt{allocate} clause will directly specify 
the use of an allocator for any construct that accepts data sharing clauses.
It enables the allocation of \texttt{private} variables in a particular memory
space. Figure~\ref{fig:allocators} illustrates the use of the pre-defined
\texttt{omp\_high\_bw\_mem\_alloc} allocator to allocate memory from the 
high bandwidth memory space.

\begin{figure}
\begin{minted}{c}
double * A = (double * ) 
    omp_alloc(N, &omp_high_bw_mem_alloc);
\end{minted}
\caption{High-Bandwidth Memory Allocation\label{fig:allocators}}
\end{figure}

In order to support rapid adaption of existing programs to a specific memory 
configuration, the pre-defined allocators are of type 
\texttt{omp\_allocator\_t}. Thus, they can be passed by argument and once 
memory allocation uses the OpenMP API function, these code places do not 
have to be modified again just to use a different memory space.
