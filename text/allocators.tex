\subsection{Allocators and Hierarchical Memory}
\label{sub:allocators_and_hierarchical_memory}

%% Christian: this is meant as a first draft for review by Alex

Modern computing devices employ a memory hierarchy, which is expected to become
even deeper in future systems, following the introduction of high-bandwidth
memory, non-volatile RAM and other developments.
Each of these technologies has a different programming interface and distinct
performance characteristics.
Addressing these differences becomes increasingly critical for all types of
computing where performance is tied to memory bandwidth: as the fastest
resources typically have limited capacity, placement choices have direct
influence on application performance.
To enable programmability of these technologies and portability across
platforms, OpenMP~5.0 includes a consistent and portable interface for memory
placement in tiered memory systems.

%% Christian to Alex: I used the term "memory space" by intent over an
%% alternative such as "memory resource", because we might want to come
%% back using that term. What do you think?

The term \emph{memory space} refers to a memory resource available in the
system, at the time the OpenMP program is executed.
Memory spaces differ in their characteristics, for instance in bandwidth or
capacity.
An \emph{allocator} is an object that allocates and frees memory from a suitable
memory space and OpenMP offers a set of pre-defined memory allocators.
The use of one of these allocators expresses the intent to allocate memory from
a memory space with a certain property.

\begin{figure}
\begin{minted}{c}
double *A = (double*) omp_alloc(N,
            &omp_high_bw_mem_alloc)
\end{minted}
\caption{Allocation from high-bandwidth memory.\label{fig:allocators}}
\end{figure}

The pre-defined memory allocators support the selection of a memory space with
large capacity, high bandwidth or low latency, or memory local to a particular
thread or thread team, for example.
The concept of memory allocation in OpenMP is supported by the
\texttt{omp\_alloc} and \texttt{omp\_free} routines, which are offered as direct
replacements for \texttt{malloc} and \texttt{free} from the standard library.
An allocator can also be used within the \texttt{allocate} clause that is
supported on all constructs that accept data sharing clauses.
It enables the allocation of \texttt{private} variables in a particular memory
kind.
Figure~\ref{fig:allocators} illustrates the use of the pre-defined
\texttt{omp\_high\_bw\_mem\_alloc} allocator to allocate memory from the memory
kind with high bandwidth.

%% Christian to Alex: do you think we need to provide the full list here?
%% I have no idea about how much space we have, but Bronis said something
%% like one page for all additions, this is why I did not include it yet.

In order to support the outfitting of existing programs and to enable the quick
adoption of a program to a specific memory configuration, the pre-defined
allocators are of type \texttt{omp\_allocator\_t}.
This way they can be passed by argument and once the memory allocation has
been changed to using the OpenMP API function, these code places do not have to
be modified again just for using a different memory kind.
