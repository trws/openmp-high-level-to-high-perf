\subsection{Allocators and Hierarchical Memory}
\label{sub:allocators_and_hierarchical_memory}

Modern computing devices employ a memory hierarchy, which is expected to become
even deeper in future systems, following the introduction of high-bandwidth
memory, non-volatile RAM and other developments.
Each of these technologies has a different programming interface and distinct
performance characteristics.
Addressing these differences becomes increasingly critical for all types of
computing where performance is tied to memory bandwidth: as the fastest
resources typically have limited capacity, placement choices have direct
influence on application performance.
To enable programmability of these technologies and portability across
platforms, OpenMP~5.0 includes a consistent and portable interface for memory
placement in tiered memory systems.

The term \emph{memory space} refers to a memory resource available in the
system, at the time the OpenMP program is executed. Memory spaces differ in their characteristics, for instance in bandwidth or
capacity. OpenMP defines a set of pre-defined \emph{memory spaces} that map to memory resources found in today's HPC systems.

An \emph{allocator} is an object that is associated to a memory space when created. It allows to allocate and free memory from the resources of its associated memory space. OpenMP offers a set of pre-defined memory allocators that match the pre-defined memory spaces. The pre-defined memory allocators support, for example, the selection of a memory space with large capacity, high bandwidth or low latency, or memory local to a particular thread or thread team.

\begin{figure}
\begin{minted}{c}
double *A = (double*) omp_alloc(N,
            omp_high_bw_mem_alloc)
\end{minted}
\caption{Allocation from high-bandwidth memory.\label{fig:allocators}}
\end{figure}

The concept of memory allocation in OpenMP is supported by the
\texttt{omp\_alloc} and \texttt{omp\_free} routines, which are offered as supersets for \texttt{malloc} and \texttt{free} from the C standard library.
An allocator can also be used within the \texttt{allocate} clause that is
supported on all constructs that accept data sharing clauses.
It enables the allocation of \texttt{private} variables in a particular memory
kind.
Figure~\ref{fig:allocators} illustrates the use of the pre-defined
\texttt{omp\_high\_bw\_mem\_alloc} allocator to allocate memory from the memory
kind with high bandwidth.

In order to support the outfitting of existing programs and to enable the quick adoption of a program to a specific memory configuration, the allocators are of type \texttt{omp\_allocator\_t *} and they can be treated as regular pointers in the program.
This allows to pass them by argument and once the memory allocation calls have been changed to using the OpenMP API functions, these code places do not have to be modified again just to use a different memory kind but only the allocator passed to the function needs to be adjusted.

\begin{figure}[htb]
\begin{minted}[fontsize=\footnotesize]{c}
void some_function ( omp_allocator_t *allocator )
{
   double some_array[N];
#pragma omp parallel private(some_array) \
                     allocate(allocator:some_array)
   {
       ...
   }
}

some_function(omp_high_bw_mem_alloc);
some_function(omp_default_mem_alloc);
\end{minted}
\caption{Separating memory configuration from allocations.\label{fig:separation-concerns-alloc}}
\end{figure}

Figure~\ref{fig:separation-concerns-alloc} illustrates how a function can be defined to allow callers to define the memory policy the function should use when allocating the private array \emph{some\_array}.

In addition to pre-defined allocators OpenMP also offers the possibility of creating custom memory allocators where the user can further specify some traits of the allocator that change its behavior. Current traits allow to specify the desired memory alignment, the maximum pool size, the fallback behavior when failing to allocate memory and some hints that allow programmers to specify who will use the allocated memory or expected contention on the allocator.

\begin{figure}[hbt]
\begin{minted}[fontsize=\footnotesize]{c}
omp_alloctraits_t *traits[]=
                 {{OMP_ATK_ALIGNMENT,64},
                  {OMP_ATK_ACCESS,OMP_ATV_THREAD}};
omp_allocator_t *allocator = 
          omp_init_allocator(omp_default_mem_space,
                             2,traits);
\end{minted}
\caption{Creating a custom memory allocator.\label{fig:custom-allocator}}
\end{figure}

Figure~\ref{fig:custom-allocator} shows an example of how a custom allocator can be created in OpenMP~5.0. This particular allocator will return memory from the \emph{default memory space} with 64-byte alignment and the memory will only be accessible by the thread that does the allocations. This allocator can then be used in the previously presented API calls, directives and clauses.
