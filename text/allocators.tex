\subsection{Memory Allocation}
\label{sub:allocators_and_hierarchical_memory}

Memory hierarchies will become deeper in future systems with the use of 
technologies such as high-bandwidth memory and non-volatile RAM. Each 
of these technologies has a different programming interface and distinct
performance characteristics. Programming mechanisms must address these 
differences and support intelligent data placement since the fastest resources
typically have limited capacity. To enable programmability of these 
technologies and portability across platforms, OpenMP~5.0 will include a 
consistent and portable interface for placement within the memory hierarchy.

A \emph{memory space} is a memory resource that is available in the system.
Memory spaces differ in their characteristics, for instance in bandwidth or 
capacity. OpenMP will define intuitive pre-defined memory spaces that map to 
memory resources in HPC systems. An \emph{allocator} object allocates and 
frees memory from the resources of the memory space to which it is associated 
when it is created. OpenMP~5.0 will provide pre-defined memory allocators that
match its pre-defined memory spaces. For example, the pre-defined memory 
allocators can select a memory space with large capacity, high bandwidth or 
low latency, or local to a particular thread or thread team.

\begin{figure}
\begin{minted}{c}
double * A = (double * ) omp_alloc(N, 
                omp_high_bw_mem_alloc);
\end{minted}
\caption{High-Bandwidth Memory Allocation\label{fig:allocators}}
\end{figure}

OpenMP 5.0 will include the \texttt{omp\_alloc} and \texttt{omp\_free} 
routines as supersets for \texttt{malloc} and \texttt{free}. The 
\texttt{allocate} directive can specify allocation properties of variables 
that are not allocated through an API call such as global or stack variables. 
The \texttt{allocate} clause will directly specify the use of an allocator 
for any construct that accepts data sharing clauses. It enables the allocation
of \texttt{private} variables in a particular memory space. 
Figure~\ref{fig:allocators} illustrates the use of the pre-defined 
\texttt{omp\_high\_bw\_mem\_alloc} allocator to allocate memory from the 
high bandwidth memory space.

In order to support rapid adaption of existing programs to a specific memory 
configuration, the pre-defined allocators have type 
\texttt{omp\_allocator\_t *} and can be used as regular pointers. Thus, they 
can be passed by argument and once memory allocation uses the OpenMP API 
function, these code places do not have to be modified again just to use a 
different memory space; the allocator passed to the function only needs to 
be adjusted. Figure~\ref{fig:separation-concerns-alloc} illustrates how to
select the memory policy that a function used to allocate the private array 
\emph{some\_array}.


\begin{figure}[t]
\begin{minted}[fontsize=\footnotesize]{c}
void some_function (omp_allocator_t * allocator) {
  double some_array[N];
  #pragma omp parallel private(some_array)        \
                     allocate(allocator:some_array)
  {
     ...
  }
}

some_function(omp_high_bw_mem_alloc);
some_function(omp_default_mem_alloc);
\end{minted}
\caption{Separate Memory Selection and Allocation\label{fig:separation-concerns-alloc}}
\end{figure}

Besides pre-defined allocators, OpenMP~5.0 will support creation of custom 
memory allocators through which the user can specify additional traits.
Current traits can specify the desired memory alignment, 
the maximum pool size, the fallback behavior when failing to allocate memory 
and some hints that specify the context in which the memory is expected to be
used or the expected contention on the allocator. 
Figure~\ref{fig:custom-allocator} shows an example that creates a custom 
allocator. This allocator returns memory from the \emph{default memory space} 
with 64-byte alignment that only the thread that allocates the memory can 
access. This allocator can then be used in the previously presented API calls, 
directives and clauses.

\begin{figure}[t]
\begin{minted}[fontsize=\footnotesize]{c}
omp_alloctraits_t *traits[]=
                 {{OMP_ATK_ALIGNMENT,64},
                  {OMP_ATK_ACCESS,OMP_ATV_THREAD}};
omp_allocator_t *allocator = 
          omp_init_allocator(omp_default_mem_space,
                             2,traits);
\end{minted}
\caption{Custom Memory Allocator\label{fig:custom-allocator}}
\end{figure}


