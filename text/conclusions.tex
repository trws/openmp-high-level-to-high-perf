\section{Conclusion}
\label{sec:conclusion}

Over twenty years have passed since we released the first OpenMP specification.
It has become a mature programming API that continues to support Fortran, C, and
C++ as base languages. In its maturation, the size of the API and its
specification has grown substantially as we added support for additional
parallel programming patterns. Its underlying philosophy has also evolved
although we retain many of its core principles. Most of all, the primary purpose
of the API continues to be to allow users to specify information about their
computation that they easily know but that would require complex compiler
analysis to deduce while relying on the compiler to implement repetitive,
tedious and error-prone mechanisms that exploit that information in a way that
can be carried from compiler to compiler.  As of this writing, the OpenMP 
compilers page~\cite{openmp-compilers} lists sixteen compilers, nine of 
which support at least a significant portion of OpenMP 4.5.

In this paper, we discussed the $7.5\times$ increase in the size of the 
OpenMP specification over the course of its lifetime. We provided 
a glimpse into the evolution of its guiding principles as well as
some of the features that the most recent versions added. We also
discussed some of the key programming features that OpenMP 5.0 will
add and that are under consideration for versions beyond it. These
plans will result in a specification that supports essentially every 
major parallel programming pattern and the latest base language standards.


