\subsection{Enabling Language-Level Outlining}
\label{sub:enabling_language_level_outlining}

As we discussed in Section~\ref{sub:outlining} outlining, or extraction 
of code into functions by the compiler, is one of the core mechanisms 
used to implement OpenMP. Some base languages provide outlining mechanisms 
in the form of closures or lambdas. The writers of libraries and parallel 
frameworks find these constructs attractive since they can describe abstract 
patterns and behaviors that then are passed an arbitrarily complex code 
sequence and associated data. Frameworks like Kokkos~\cite{kokkos} and 
RAJA~\cite{raja} exploit this mechanism to create flexible looping constructs, 
like the one in Figure~\ref{fig:raja}, that can be compiled for host devices, 
targets or any number of parallel backends depending on compile time arguments.
These mechanisms pose challenges to OpenMP, which OpenMP 5.0 will begin to
address.

\begin{figure}
\begin{minted}{c}
RAJA::forall<omp_parallel_for>(
    RAJA::range(0, n),
    [=](int a) {
      // loop body
});
\end{minted}
\caption{A \texttt{parallel for} loop body in a C++ lambda.\label{fig:raja}}
\end{figure}

While OpenMP must evolve to support mechanisms such as lambdas, Fortran
users of OpenMP currently cannot exploit the capabilities that we will 
provide. So far, although many OpenMP implementations use outlining, they 
do not expose the resulting functions to the user. However, exposing these 
functions could provide many benefits, including a mechanism to support
closures in Fortran.
 
We could extend the \texttt{task} directive to create a form 
of "callable task" or OpenMP closure object that would be portable across 
C, C++ and Fortran. The extension would significantly reduce the work 
required to make an arbitrary callable object with state in C and Fortran.
It would also support library implementations with functionality like that 
of Kokkos and RAJA that all three languages could use. Challenges remain,
however, particularly how to integrate the functionality well into OpenMP 
and how to make it as efficient as possible at runtime. A simple and 
portable solution generates a structure, or derived type, and a function 
pointer. It would integrate well with established libraries, but is unlikely 
to optimize well for constructs that will be called many times. Despite 
the challenges, giving users control of outlining could be
a major step forward for OpenMP.


