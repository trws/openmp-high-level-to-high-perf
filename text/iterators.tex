\subsection{Iterators}
\label{sub:iterators}

One emerging need in the usage of OpenMP is the ability to increase the dynamic
nature of OpenMP description (its effects are mostly dynamic). For example, the
depend clause allows to express that a task or a target construct has
a dependence on different elements of an array. You can express this today by
writing multiple clauses which can easily become tedious and error prone if it
is more than a few clauses. Furthermore, if the number of clauses would depend
on a runtime decision, like if this is a corner cell or an inner cell, it would
be impossible to describe in a single directive. The same reasoning could be
applied to the map clause in target directives where you would like to describe
different elements of an array.
   
To overcome this lack of expressiveness OpenMP iterators are being developed.
They allow to iterate through a range of values and evaluate the same clause
multiple times with each value in the range. Furthermore, this range can be
described at runtime which allows to vary both the number clauses and their
content for a given directive. Figure~\ref{fig:iterators} shows how the iterator
feature can be used to express a variable number of dependences on a \emph{task}
construct.

\begin{figure}
\begin{minted}{c}
void func( double *a, int N )
{
#pragma omp task depend(inout:a[i]:i=0:N)
   work(a);
}
\end{minted}
\caption{Example of task dependencies with iterators}
\label{fig:iterators}
\end{figure}
   
One could see this as first step towards providing OpenMP with metaprogramming
capabilities that would allow were possibly to construct the directives either
at compile time or runtime instead of only allowing them to be statically typed
by a programmer.


   

