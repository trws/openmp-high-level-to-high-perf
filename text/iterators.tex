\subsection{Iterators}
\label{sub:iterators}

Many OpenMP clauses accept lists of parameters. In OpenMP 4.5 or earlier, 
while many OpenMP clauses accept expressions, the expressions (but not 
their values) had to be fully determined at compile time. Thus, the number 
of elements of the lists for the clauses is static. As a result, for example,
the \texttt{depend} clause can specify a dependendence on multiple elements
of an array but the number of elements (or array sections) must be known
at compile time. This requirement can prevent the expression of some
algorithms or make their expression more complex.  For example, if a 
corner cell has fewer dependences than an inner cell then the user may need 
to modify the base language code to provide separate annotations for the
different situations. Further, the limitation can require the use of long 
error-prone lists even when the number of list elements is static. This 
limitation arises from the lack of general programming constructs in 
OpenMP directives, which we plan to reduce as we discussed in 
Section~\ref{sub:relationship_to_base_languages}, 

\begin{figure}
\begin{minted}{c}
void func( double *a, int N )
{
#pragma omp task depend(inout:a[i]:i=0:N)
   work(a);
}
\end{minted}
\caption{Iterated Task Dependences}
\label{fig:iterators}
\end{figure}

To overcome this lack of expressiveness, OpenMP will add the concept of  
iterators. This mechanism can iterate through a range of values. Thus, a
clause can have a dynamic number of list elements. Figure~\ref{fig:iterators} 
shows how the iterator feature can express a \texttt{task} construct with a 
variable number of dependences. 
   


   

