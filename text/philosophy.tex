\section{The Guiding Philosophy of OpenMP}
\label{sec:philosophy}

The general philosophy of OpenMP reflects the ARB's mission to standardize 
directive-based multi-language high-level parallelism that is performant, 
productive and portable. Portability is achieved first and foremost through
broad adoption and support of the specification. At the highest level, a 
directive-based approach supports productivity through incremental 
parallelization and refinement through which user code remains as close 
to its original serial version as possible while still achieving performance 
goals. Directives allow the programmer to specify information that a compiler
would otherwise not be able to determine -- or that might require complex
and error-prone analysis. Further, OpenMP provides sensible defaults that 
often result in high performance but also allows low-level control of 
aspects that the compiler and runtime may not deduce high quality settings.

As we discussed in the introduction, OpenMP retains many of its original
goals, which embodied a general philosophy. However, like the specification, 
the philosophy of OpenMP has evolved as it has expanded to support a wider
range of parallel programming patterns. The remainder of this section 
discusses the evolution of two key aspects of the original philosophy,
language independence and serial equivalence, as well as the issue of
descriptiveness versus prescriptiveness, a philosophical issue for 
programming models that has recently received significant attention.

\subsection{Relationship to Base Languages}
\label{sub:relationship_to_base_languages}

Although OpenMP began with separate specifications for C/C++ and 
Fortran, as we discussed in Section~\ref{sec:evolve}, OpenMP 2.5 
merged them into a single document. Although that choice was
partly due to the pragmatic reduction in effort to move the base 
languages forward together, the original goal of a consistent API 
across the base languages, which remains a key part of OpenMP's 
guiding philosophy, was the most important reason. This language 
independence is one of OpenMP's core strengths since OpenMP has 
greater portability and generality, not only across the C, C++ 
and Fortran base languages but also in its design as a result. 

%% BRONIS: My solution is to merge the two sections since
%% BRONIS: language independence is just one aspect of the
%% BRONIS: relationship to the base language
%% \tom{this is too short, ideas on how to expand?}

%% BRONIS: Still need to rework the following paragraph
%% BRONIS: Need to introduce the concept of directives
%% BRONIS: being Turing complete and that OpenMP can bridge
%% BRONIS: differences across the languages

OpenMP, by itself, is not a language.  It provides an API for expressing
parallelism and concurrency in a portable way across three independent
languages, with the goal of providing the same experience and easy
interoperability between all three.  Because of this it relies heavily on each
base language to define the behavior of a given construct within each thread of
execution or block of code.  The relationship with the base languages has
changed somewhat over time however.  Before the release of C11 and C++11, C and
C++ themselves had no well-defined concept of a data race, or of threading in
general.  In fact, the ISO C99 standard~\cite{c99} does not contain the word
"thread" at all, and contains the word "race" only as part of the word "brace."
As a result, OpenMP has to provide all threading and memory model semantics for
a program written using OpenMP constructs in the context of a C99 program.  With
the advent of integrated threading models, acquire and release memory models and
more built-in parallel concepts OpenMP is now in a position of providing its own
semantics in the context of those models and integrating with them.



\subsection{Serial Equivalence}
\label{sub:serial_equivalence}

A original goal for OpenMP was to support serial equivalence to the 
greatest possible extent. As a result, many think that all OpenMP programs, 
or at least all correct OpenMP programs, are guaranteed the same result
if the code is executed in parallel as when the compiler completely 
ignores all OpenMP constructs. However, even OpenMP 1.0 included runtime
functions that allow a program to depend on the number of threads or the
thread number (or ID) that executed a region. Thus, trivial programs could
fail to exhibit serial equivalence. Today, many more opportunities exist
to write OpenMP programs that do not provide serial equivalence. 
%% BRONIS:Could include the trivial example here

%% BRONIS: Tom, you mentioned data privatization in the text that I
%% BRONIS: deleted. Do you have an interesting example in mind?

As OpenMP has evolved, the opportunities to write programs that do not
exhibit serial equivalence have increased. OpenMP support for tasking 
provides numerous opportunities. Figure~\ref{fig:trivial_task} provides
a simple tasking program in which the serial version has an infinite loop
while the parallel version will complete quickly, assuming that the parallel
region uses two or more threads and different threads execute the two tasks.
Figure~\ref{fig:trivial_target} shows a pair of example functions, where
\texttt{increment} will take and increment a value whether it executes with
OpenMP or not, \texttt{example} may or may not depending on the platform even
though it calls \texttt{increment} to do the work.  While this example clearly
contains a bug, such manipulations of device data lifetimes can cause different
execution between OpenMP and serial \emph{even when the OpenMP code is also
serial}.

\begin{figure}
\begin{minted}{c}
void example() {
  int a = 0;
  #pragma omp parallel
  {
    #pragma omp single
    {
      int b = 0;
      #pragma task
      while (b == 0) {
        #pragma atomic read seq_cst
        b = a;
      }
      #pragma task
      {
        #pragma atomic update seq_cst
        a++;
      }
    }
  }
}
\end{minted}
\caption{A trivial program with OpenMP tasking.\label{fig:trivial_task}}
\end{figure}

\begin{figure}
\begin{minted}{c}
void increment(int * p) {
  #pragma omp target map(tofrom:p[0:1])
  {
    (*p)++;
  }
}
void example() {
  #pragma omp target data map(to: p[0:1])
  example1();
}
\end{minted}
\caption{A trivial program with OpenMP target.\label{fig:trivial_target}}
\end{figure}

In general, serial equivalence requires the program or runtime to limit
the possible execution orders. As OpenMP has grown to support more parallel 
programming patterns, the range of execution orders has as well, which 
implies either more opportunities to eschew serial equivalence or more
limitations on the range of execution orders. Since such limitations would 
also create performance limitations, OpenMP (or any parallel programming 
system) tries to avoid them unless the programmer requires them. Thus, the 
philosophy of OpenMP remains to provide constructs that simplify the 
enforcement of serial equivalence when desired but that has evolved not 
to limit parallelism -- and by extension -- performance unnecessarily.


\subsection{Descriptive or Prescriptive Semantics}
\label{sub:descriptive}

The high performance community is currently debating the value of 
\emph{descriptive} versus \emph{prescriptive} programming semantics. 
Semantics are descriptive if programming constructs describe the 
computation that should be performed but provide the compiler and 
runtime the flexibility to determine exactly how to perform the 
computation. Programming constructs with prescriptive semantics 
prescribe all details of how to perform the required computation.

Our position is that the debate is misguided since it assumes a binary 
choice between the two types of semantics. However, almost all languages 
have constructs that are descriptive while others are (more) prescriptive. 
Specifically within the HPC community, some claim that OpenACC is descriptive 
while OpenMP is prescriptive~\cite{juckeland2016isc,wolfe16descriptive}. 
While OpenACC provides more descriptive constructs in its most recent version 
than OpenMP does, the \texttt{acc parallel loop} directive is prescriptive 
since sometimes users want to \emph{prescribe} that a loop must be parallelized. 

Alternatively, most OpenMP defaults allow the compiler freedom
to choose details about how the computation is performed. Even
the \texttt{num\_threads} clause of the \texttt{parallel} construct,
which many believe to be among its most prescriptive mechanisms,
allows the compiler and runtime to determine if the number of
threads requested are available. If that many threads are not
available, the compiler and runtime have the flexibility to
determine how many threads to use. So, one may see the issue 
as where to place a language, or even its constructs, on a 
continuum of possible semantics.

More importantly, choosing one place on that continuum is overly
limited and fails to address the overall preference of programmers. 
Specifically, they would prefer that the compiler and runtime would 
always ``do the right thing'' given a description of the computation 
to perform. However, in reality, compilers and runtimes often do not. 
In these instances, programmers prefer to have the ability to override 
their decisions and to prescribe exactly how to perform the computation.

For these reasons, the emerging OpenMP philosophy is to provide
mechanisms that describe the computation to perform and that
prescribe as much or as little as the programmer desires about
how to perform it. As a first step, OpenMP 5.0 will add the 
\texttt{loop} construct, which only informs the compiler 
and runtime that a loop nest is easily parallelized. In the 
longer run, we are exploring mechanisms that specify that the 
intent of a clause or a construct is fully descriptive or prescriptive.

