\section{The Guiding Philosophy of OpenMP}
\label{sec:philosophy}

The general philosophy of OpenMP reflects the ARB's mission to standardize 
directive-based multi-language high-level parallelism that is performant, 
productive, and portable. Portability is achieved first and foremost through
broad adoption and support of the specification. At the highest level, a 
directive-based approach supports productivity through incremental 
parallelization and refinement through which user code remains as close 
to its original serial version as possible while still achieving performance 
goals. Directives allow the programmer to specify information that a compiler
would otherwise not be able to determine -- or that might require complex
and error-prone analysis. 

\textcolor{red}{
OpenMP provides sensible defaults that often result in high
performance but also allows low-level control of aspects for which the compiler
and runtime may not deduce high-quality settings. Programmers can thus start
from a very simple usage of OpenMP directives and incrementally increase the
level of complexity to expose more and more control over the code
transformations applied and parallel execution to achieve higher performance.
Despite the (growing) complexity of OpenMP directives, the OpenMP language is
designed to maintain this core principle of directives building on top of each
other to support this incremental program evolution.
}

As we discussed in the introduction, OpenMP retains many of its original
goals, which embodied a general philosophy. However, like the specification, 
the philosophy of OpenMP has evolved as it has expanded to support a wider
range of parallel programming patterns. The remainder of this section 
discusses the evolution of two key aspects of the original philosophy,
language independence and serial equivalence, as well as the issue of
descriptiveness versus prescriptiveness, a philosophical issue for 
programming models that has recently received significant attention.

\subsection{Relationship to Base Languages}
\label{sub:relationship_to_base_languages}

Although OpenMP began with separate specifications for C/C++ and 
Fortran, as we discussed in Section~\ref{sec:evolve}, OpenMP 2.5 
merged them into a single document. Although that choice was
partly due to the pragmatic reduction in effort to move the base 
languages forward together, the original goal of a consistent API 
across the base languages, which remains a key part of OpenMP's 
guiding philosophy, was the most important reason. This language 
independence is one of OpenMP's core strengths since OpenMP has 
greater portability and generality, not only across the C, C++ 
and Fortran base languages but also in its design as a result. 

%% BRONIS: My solution is to merge the two sections since
%% BRONIS: language independence is just one aspect of the
%% BRONIS: relationship to the base language
%% \tom{this is too short, ideas on how to expand?}

%% BRONIS: Still need to rework the following paragraph
%% BRONIS: Need to introduce the concept of directives
%% BRONIS: being Turing complete and that OpenMP can bridge
%% BRONIS: differences across the languages

OpenMP, by itself, is not a language.  It provides an API for portably
expressing parallelism and concurrency across three independent base
languages. As discussed above, OpenMP attempts to provide the same 
experience and easy interoperability between all three while also
being consistent with the specific base language. Thus, to the extent 
possible, OpenMP relies on the base language for sequential programming
constructs, including arithmetic and logical operators. However, some
mistakenly claim that a directive-based approach (i.e., pragmas in
C and C++) is necessarily limited in it scope. In reality, the approach
is Turing complete and any construct that is available in a base language 
could also be provided through a directive. 

Traditionally, OpenMP has limited its scope in several ways. However, we 
are finding that as the API grows and addresses more programming patterns, 
that we must support a larger set of basic programming constructs. As we
discuss in Section~\ref{sub:iterators}, one example is the concept of 
iterators, which provide structured looping functionality. Closures are 
another example under consideration, as we discuss in 
Section~\ref{sub:enabling_language_level_outlining}. Support for these
constructs increases the complexity of the OpenMP compilation pass so
some implementers are resisting their addition to the API. In general, 
we are currently debating the extent to which OpenMP should provide
basic programming constructs. Nonetheless, we expect the degree to
which OpenMP feels like a general programming language to increase.

%% BRONIS: I don't know if we should say something about F2015...
Regardless, OpenMP will not become a separate language and will 
continue to rely on base languages to specify the bulk of the
computation that is to be performed. It will continue to rely heavily 
on each base language to define the behavior of a given construct 
within each thread of execution or block of code. Further, we are
actively updating OpenMP to support recent base language standards. 
OpenMP 4.0 added Fortran 2003~\cite{F2003} as a normative reference
while OpenMP 5.0 will add C11~\cite{c11} and C++11~\cite{c++11} and
14~\cite{c++14} as normative references. We plan to add Fortran 
2008~\cite{F2008} soon as well.

The evolution of the base languages, as captured in their normative
references, complicates OpenMP's relationship to them. Before the 
release of C11 and C++11, C and C++ did not have any well defined 
concept of a data race, or of threading in general. In fact, the 
ISO C99 standard~\cite{c99} does not contain the term ``thread'' 
and only contains the word ``race'' as part of the term ``brace.''
In general, the original normative references did not address
parallelization. Thus, OpenMP has provided all threading and memory 
model semantics for a program that used OpenMP constructs. In 
order to provide full support for the later C and C++ standards, 
which include integrated threading models, acquire and release 
memory models and other built-in parallel concepts, OpenMP must 
ensure that its semantics do not conflict with those of the base
languages. That process has begun with TR6~\cite{openmptr6}, which is 
the most recent preview of OpenMP 5.0 and will continue beyond
OpenMP 5.0.

Finally, while a pragma-based approach is a natural one for Fortran
and C programmers, it is not the most natural one for C++. Besides
complex questions related to support for parallelism and for lambdas
that arise with the latest C++ standards, we are beginning to look at
other possible mechanisms for C++, such as attributes.



\subsection{Serial Equivalence}
\label{sub:serial_equivalence}

A common misconception about OpenMP is that all OpenMP programs, or at least all
correct OpenMP programs, provide serial equivalence. In this case serial
equivalence refers to a guarantee that if a program is compiled with all OpenMP
constructs completely ignored it will get the same answer as both the original
serial code and the OpenMP code in parallel.  The truth is significantly more
complicated.  All OpenMP constructs make it possible to write a program with
serial equivalence, but it is equally possible to write correct, well-formed
programs that get different answers not only between sequential and parallel but
also for any given number of threads.

The reason for this is that OpenMP provides information to the user about how
their code is being run and attempts to offer high performance by default, and
serial equivalence if needed.  It is possible to write code that depends on the
number of threads by using runtime routines to query the thread count and the
identity of a given thread.  Given code that avoids doing anything based on the
number of threads, there are still two factors that can create programs without
serial equivalence: data privatization and execution order.  For example, a
workshared loop with a reduction clause makes no guarantee of the order in which
the intermediate products are combined.  This makes it both non-deterministic
and non-serial-equivalent for any reduction over floating point numbers.

Serial equivalence will continue to be possible in OpenMP for the foreseeable
future, but as the specification grows and more control is given to the user,
the number of ways to write code without serial equivalence also grows.


\subsection{Descriptive or Prescriptive Semantics}
\label{sub:descriptive}

A debate has emerged within the high performance language
community about whether programming languages should 
provide \emph{descriptive} or \emph{prescriptive} semantics. 
A language provides descriptive semantics if its constructs 
describe the computation that should be performed but provides 
the compiler and runtime the flexibility to determine exactly 
how the computation is performed. Alternatively, prescriptive
semantics not only describe the required computation but 
also all details of how to perform it.

Our position is that the debate is misguided for two reasons.
First, the debate assumes a binary choice between the two 
types of semantics. However, almost all languages have constructs 
that are descriptive while others are (more) prescriptive. 
%% BRONIS: Perhaps we could find one of Michael Wolfe's 
%% BRONIS: HPCWire articles that makes this claim?
Specifically within the HPC community, the claim has been 
made that OpenACC is descriptive while OpenMP is prescriptive.
For example, the BLAH construct of OpenACC...
Alternatively, most OpenMP defaults allow the compiler freedom
to choose details about how the computation is performed. Even
the \texttt{num\_threads} clause of the \texttt{parallel} construct,
which many believe to be among its most prescriptive mechanisms,
allows the compiler and runtime to determine if the number of
threads requested are available. If that many threads are not
available, the compiler and runtime have the flexibility to
determine how many threads to use. So, one may see the issue 
as where to place a language, or even its contructs, on a 
continuum of possible semantics.

Second, even the view of placement on a continuum is overly
limited and fails to address the overall preference of 
programmers. Specifically, they would prefer that the compiler
and runtime would always ``do the right thing'' given a 
description of the computation to perform. However, in reality,
compilers and runtimes often do not. In these instances,
programmers prefer to have the ability to override the
decisions and to specifiy exactly how to perform the computation.

For these reasons, the emerging OpenMP philosophy is to provide
mechanisms that describe the computation to perform and that
prescribe as much or as little as the programmer desires about
how to perform it. As a first step, OpenMP 5.0 will add the 
\texttt{concurrent} construct, which only informs the compiler 
and runtime that a loop nest is easily parallelized. In the 
longer run, we are exploring mechanisms that specify that the 
intent of a clause or a construct is fully descriptive or prescriptive.

