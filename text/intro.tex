\section{Introduction}
\label{sec:intro}

The OpenMP effort began in 1996 when a handful of vendors (DEC, HP, IBM, Intel,
Kuck and Associates, and SGI) were brought together by the Accelerated Strategic
Computing Initiative~(ASCI) of the Department of Energy~(DOE) to create a
portable application programming interface~(API) for shared memory computers
based on their various implementations of, and extensions to, the Parallel
Computing Forum directives~\cite{TheParallelComputingForum}. Vendors do not
typically work well together unless an outside force compels cooperation. Mary
Zosel and the ASCI parallel tools team provided that compulsion by communicating
that ASCI would only purchase systems with a portable API for shared memory
programming. Their role in the beginning of OpenMP ensured that it met the needs
of HPC applications programmers.

Early public presentations about the project~\cite{ewomp99} clearly
defined the initial group's goals:
%The OpenMP Architecture Review Board and the future of OpenMP, Tim Mattson,
%European Workshop on OpenMP, Lund, Sweden, September 30 - October 1, 1999

\begin{itemize}
  \item To support portable, efficient and comprehensible shared-memory 
        parallel programs;
  \item To produce specifications based on common practice that 
        could be readily implemented;
  \item To provide a consistent API for Fortran, C and C++ to the 
        most reasonable extent possible;
  \item To be lean and mean, i.e., to  be only as large as required 
        to express important control-parallel, shared-memory programs  
        but no larger;
  \item To ensure API versions are backwards compatible;
  \item To support \emph{serial equivalence}, i.e., for it to be possible to
    write OpenMP programs to produce the same result whether run serially or in
        parallel, to the greatest possible extent.
\end{itemize}

The first OpenMP specification  was released in November 1997 at SC97. The
early OpenMP community knew that other parallel programming  standardization 
efforts, such as HPF~(High Performance Fortran) and MPI 2.0, suffered from 
multi-year delays as implementors struggled to produce robust, 
application-ready implementations. Thus, OpenMP by design narrowly focused 
on current practice. This focus led to the availability of multiple
vendor-supported implementations within a year of the release of the 
first specification. 

\begin{figure}
\begin{minted}{c}
#pragma omp parallel // fork
{
  // share the loop across threads,
  // reducing into total
  #pragma omp for reduction(+:total)
  for (int i=0; i<N; ++i) {
    total += foo(i);
  }
} // join
\end{minted}
\caption{Basic OpenMP\label{fig:basic}}
\end{figure}

Over time, additional vendors and research organizations joined the effort.  A
non-profit corporation, the OpenMP Architecture Review Board~(ARB), was created
to prevent any single vendor from dominating the standard. The current 30
members of the OpenMP ARB continue to own and to evolve the API to serve the
needs of parallel application programmers. The ARB retains many of the original
goals in its current mission, which is to standardize directive-based
multi-language high-level parallelism that is performant, productive and
portable. The OpenMP API now provides a portable, scalable model that gives
parallel programmers a simple and flexible interface for developing parallel
applications for platforms ranging from embedded systems and accelerator
devices to multicore systems.  A simple example of the iconic workshared
parallel for loop is shown in Figure~\ref{fig:basic}, showing the basic
fork-join model of parallel with a workshared loop and a reduction in OpenMP
syntax that has been valid from the first release of the C version of the
specification until today.

OpenMP retains all but two of its original goals. Specifically, OpenMP has 
evolved to support almost all parallel programming patterns, which necessarily
implies a larger specification than originally envisioned. Further, while 
serial equivalence is still achievable, that range of patterns necessarily 
leads to many opportunities to deviate from it. Otherwise, the only change 
to the original goals is that the scope of OpenMP has extended beyond 
shared memory. 

We comprehensively examine the state of OpenMP in anticipation of the imminent 
release of version 5.0 of the API. We first review the evolution of OpenMP 
through version 4.5 (Section~\ref{sec:evolve}). We then provide a more 
detailed examination of the philosophy that has guided its evolution 
(Section~\ref{sec:philosophy}). Next, we briefly review the basic concepts 
and mechanisms that support implementation of the evolving API 
(Section~\ref{sec:concepts}). We then detail some recent 
(Section~\ref{sec:recent_extensions}) and impending 
(Section~\ref{sec:in_progress}) additions to OpenMP. Finally, we discuss
and anticipate some possible directions for its longer term evolution
(Section~\ref{sec:future_directions}).
