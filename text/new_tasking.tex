\subsection{Further Evolution of Tasking Support}
\label{sub:new_tasking}

OpenMP 5.0 continues to evolve the tasking model to address use cases. Task 
reductions, task affinity, and additional forms of task 
dependences enhance performance and ease of use. Prior to OpenMP 5.0, lack 
of support for explicit task reductions required users to implement their 
own reductions by collecting and later combining per-thread partial values, 
passing partial values through the tree of tasks, or using locks or atomics 
that serialize those operations. The \texttt{task\_reduction} clause allows 
a reduction over a task group, and the \texttt{reduction} clause is available 
on task loops. The \texttt{in\_reduction} clause appears on tasks that 
participate in the reduction, which can include target tasks that 
offload computation or transfer data to devices.

Support for task dependences is extended in two new ways.  First, use 
of iterators is allowed in the \texttt{depend} clause, as described 
previously in Section~\ref{sub:iterators}. Second, a new dependence type 
allows a set of tasks to commute with respect to one another with the 
constraints that their executions are mutually exclusive and that 
they satisfy any dependences with respect to tasks outside the set.

Like task dependences, task affinity indicates the data used by a task. 
However, task affinity is a hint that guides the scheduling of tasks to 
threads, rather than enforcing an ordering among the threads. Tasks that 
use the same data can be scheduled to the same thread or to threads that 
execute on cores in the same NUMA domain. An advanced runtime may also use 
the information to tune work stealing for better locality. Future versions 
of OpenMP may apply the \texttt{affinity} clause to other constructs besides 
the \texttt{task} construct.

