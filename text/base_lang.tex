\subsection{Relationship to Base Languages}
\label{sub:relationship_to_base_languages}

Although OpenMP began with separate specifications for C/C++ and 
Fortran, as we discussed in Section~\ref{sec:evolve}, OpenMP 2.5 
merged them into a single document. Although that choice was
partly due to the pragmatic reduction in effort to move the base 
languages forward together, the original goal of a consistent API 
across the base languages, which remains a key part of OpenMP's 
guiding philosophy, was the most important reason. This language 
independence is one of OpenMP's core strengths since OpenMP has 
greater portability and generality, not only across the C, C++ 
and Fortran base languages but also in its design as a result. 

%% BRONIS: My solution is to merge the two sections since
%% BRONIS: language independence is just one aspect of the
%% BRONIS: relationship to the base language
%% \tom{this is too short, ideas on how to expand?}

%% BRONIS: Still need to rework the following paragraph
%% BRONIS: Need to introduce the concept of directives
%% BRONIS: being Turing complete and that OpenMP can bridge
%% BRONIS: differences across the languages

OpenMP, by itself, is not a language.  It provides an API for expressing
parallelism and concurrency in a portable way across three independent
languages, with the goal of providing the same experience and easy
interoperability between all three.  Because of this it relies heavily on each
base language to define the behavior of a given construct within each thread of
execution or block of code.  The relationship with the base languages has
changed somewhat over time however.  Before the release of C11 and C++11, C and
C++ themselves had no well-defined concept of a data race, or of threading in
general.  In fact, the ISO C99 standard~\cite{c99} does not contain the word
"thread" at all, and contains the word "race" only as part of the word "brace."
As a result, OpenMP has to provide all threading and memory model semantics for
a program written using OpenMP constructs in the context of a C99 program.  With
the advent of integrated threading models, acquire and release memory models and
more built-in parallel concepts OpenMP is now in a position of providing its own
semantics in the context of those models and integrating with them.


