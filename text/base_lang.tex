\subsection{Language Independence}
\label{sub:language_independence}

Though OpenMP began with separate specifications for C and Fortran, the decision
was made early on to merge the two into a single document and push both base
languages forward together.  This draws to one of the core philosophies, and
arguably core strengths, of OpenMP as a model which is its portability and
compatibility across C, C++ and Fortran.  For scientific codebases, especially
those with long lived legacy code, this can be a significant advantage.  

\tom{this is too short, ideas on how to expand?}

\subsection{Relationship to the Base Language}
\label{sub:relationship_to_the_base_language}

OpenMP, by itself, is not a language.  It provides an API for expressing
parallelism and concurrency in a portable way across three independent
languages, with the goal of providing the same experience and easy
interoperability between all three.  Because of this it relies heavily on each
base language to define the behavior of a given construct within each thread of
execution or block of code.  The relationship with the base languages has
changed somewhat over time however.  Before the release of C11 and C++11, C and
C++ themselves had no well-defined concept of a data race, or of threading in
general.  In fact, the ISO C99 standard~\cite{c99} does not contain the word
"thread" at all, and contains the word "race" only as part of the word "brace."
As a result, OpenMP has to provide all threading and memory model semantics for
a program written using OpenMP constructs in the context of a C99 program.  With
the advent of integrated threading models, acquire and release memory models and
more built-in parallel concepts OpenMP is now in a position of providing its own
semantics in the context of those models and integrating with them.


