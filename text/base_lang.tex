\subsection{Relationship to Base Languages}
\label{sub:relationship_to_base_languages}

Although OpenMP began with separate specifications for C/C++ and 
Fortran, as we discussed in Section~\ref{sec:evolve}, OpenMP 2.5 
merged them into a single document. Although that choice was
partly due to the pragmatic reduction in effort to move the base 
languages forward together, the original goal of a consistent API 
across the base languages, which remains a key part of OpenMP's 
guiding philosophy, was the most important reason. This language 
independence is one of OpenMP's core strengths since OpenMP has 
greater portability and generality, not only across the C, C++ 
and Fortran base languages but also in its design as a result. 

%% BRONIS: My solution is to merge the two sections since
%% BRONIS: language independence is just one aspect of the
%% BRONIS: relationship to the base language
%% \tom{this is too short, ideas on how to expand?}

%% BRONIS: Still need to rework the following paragraph
%% BRONIS: Need to introduce the concept of directives
%% BRONIS: being Turing complete and that OpenMP can bridge
%% BRONIS: differences across the languages

OpenMP, by itself, is not a language.  It provides an API for portably
expressing parallelism and concurrency across three independent base
languages. As discussed above, OpenMP attempts to provide the same 
experience and easy interoperability between all three while also
being consistent with the specific base language. Thus, to the extent 
possible, OpenMP relies on the base language for sequential programming
constructs, including arithmetic and logical operators. However, some
mistakenly claim that a directive-based approach (i.e., pragmas in
C and C++) is necessarily limited in it scope. In reality, the approach
is Turing complete and any construct that is available in a base language 
could also be provided through a directive. 

Traditionally, OpenMP has limited its scope in several ways. However, we 
are finding that as the API grows and addresses more programming patterns, 
that we must support a larger set of basic programming constructs. As we
discuss in Section~\ref{sub:iterators}, one example is the concept of 
iterators, which provide structured looping functionality. Closures are 
another example under consideration, as we discuss in 
Section~\ref{sub:enabling_language_level_outlining}. Support for these
constructs increases the complexity of the OpenMP compilation pass so
some implementers are resisting their addition to the API. In general, 
we are currently debating the extent to which OpenMP should provide
basic programming constructs. Nonetheless, we expect the degree to
which OpenMP feels like a general programming language to increase.

Regardless, OpenMP will not become a separate language and will 
continue to rely on base languages to specify the bulk of the
computation that is to be performed. It will continue to rely heavily 
on each base language to define the behavior of a given construct 
within each thread of execution or block of code. Further, we are
actively updating OpenMP to support recent base language standards. 
OpenMP 4.0 added Fortran 2003~\cite{F2003} as a normative reference
while OpenMP 5.0 will add C11~\cite{c11} and C++11~\cite{c++11} and
14~\cite{c++14} as normative references. We plan to add Fortran 
2008~\cite{F2008} soon as well.

The evolution of the base languages, as captured in their normative
references, complicates OpenMP's relationship to them. Before the 
release of C11 and C++11, C and C++ did not have any well defined 
concept of a data race, or of threading in general. In fact, the 
ISO C99 standard~\cite{c99} does not contain the term ``thread'' 
and only contains the word ``race'' as part of the term ``brace.''
In general, the original normative references did not address
parallelization. Thus, OpenMP has provided all threading and memory 
model semantics for a program that used OpenMP constructs. In 
order to provide full support for the later C and C++ standards, 
which include integrated threading models, acquire and release 
memory models and other built-in parallel concepts, OpenMP must 
ensure that its semantics do not conflict with those of the base
languages. That process has begun with TR6~\cite{openmptr6}, which is 
the most recent preview of OpenMP 5.0 and will continue beyond
OpenMP 5.0.

Finally, while a pragma-based approach is a natural one for Fortran
and C programmers, it is not the most natural one for C++. Besides
complex questions related to support for parallelism and for lambdas
that arise with the latest C++ standards~\cite{c++17}, we are beginning to look at
other possible mechanisms for C++, such as attributes.


