\subsection{Free-Agent Threads}
\label{sub:task_only_threads}

%% BRONIS: This section would discuss the unshackled threads idea
%% BRONIS: I think I want to call them ``free-agent threads''

Currently, only threads of the parallel region in which an explicit
OpenMP task is generated can execute that task. This limitation leads
to the unintuitive (if simple) requirement that pure tasking programs 
in OpenMP must first start a \texttt{parallel} region and then must 
ensure that only one thread executes the code that generates the tasks, 
for example, by using a \texttt{single} region. This limitation can 
restrict parallelism in more complex applications since other threads 
(resources) may be idle and available to execute the tasks.

We are exploring a concept of free-agent threads to overcome this 
limitation. The mechanism would allow any thread that is not assigned
to a team to execute any explicit task. It would fully eliminate the
limitation -- all threads could execute explicit tasks that are 
generated in the initial thread without requiring an explicit 
\texttt{parallel} region. We need to resolve many details, such as
the return values for runtime routines such as \texttt{omp\_get\_thread\_num}
when executed by a thread that is not part of the team. Since this change 
will represent a major change in the OpenMP execution model, we do not 
expect to adopt it before OpenMP~6.0.




