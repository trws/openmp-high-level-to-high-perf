\subsection{SIMD}
\label{sub:simd}

Compilers have included technology to auto-vectorize loops for many years. 
HOwever, this support has limited effectiveness for real applications because 
of the complexity of determining the potential correctness and benefit of 
vectorization (e.g., are loop iterations free of dependences). These 
limitations led almost all major compilers to include implementation-defined 
vectorization directives. While frequently spelled\texttt{ivdep}, the
semantics often subtly varied across compilers. Due to the similarity 
with the original motivation for OpenMP with respect to threading 
directives, we included explicit directives to exploit SIMD parallelism 
in OpenMP 4.0.

The \texttt{simd} directive expresses that a given loop nest has no
dependences that would prevent vectorization. The compiler can then 
vectorize the loop without performing any dependence analysis. The 
directive accepts several clauses that provide further information 
and/or restrictions to guide vectorization. The \texttt{simd} directive 
is not prescriptive as the compiler may choose not to vectorize the 
loop (essentially a vector width of one).

Loops with functions pose a particular problem to vectorization. If the 
compiler has the function definition available then it could inline it 
to vectorize the loop fully. In practice, the definition is often in a
different compilation unit. Without special treatment, the compiler can
still partially vectorize the loops by repeatedly calling the scalar 
function for each element of the vector. A more efficient solution 
generates vector variants of the functions that process multiple 
elements of the vector in a single invocation. The compiler can then
use these variants in loops annotated with the \texttt{simd} directive.

OpenMP provides the \texttt{declare simd} directive to guide generation
of vector function variants. The directive accepts several clauses that
prescribe generation of efficient variants for specific use cases so a
function may be annotated with multiple \texttt{declare simd} directives.
Other clauses generally guide generation of vector variants (e.g., the 
\texttt{uniform} clause indicates that a given argument should be a 
scalar and not a vector). The compiler can also generate other variants 
that may be useful for a specific target architecture. The simple example 
in Figure~\ref{fig:simd-example} uses the OpenMP SIMD directives.

\begin{figure}
\begin{minted}{c}

#pragma omp declare simd uniform(c)
double scale(double v, double c) {
     return v * c;
}

void example(double * v, size_t n) {
    double alpha = 0.5;
#pragma omp simd
    for (size_t i = 0; i < n; i++ )
        v[i] = scale(v[i], alpha);
}
\end{minted}
\caption{OpenMP SIMD Vectorization Example\label{fig:simd-example}}
\end{figure}




