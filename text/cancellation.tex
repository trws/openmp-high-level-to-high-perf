\subsection{Cancellation}
\label{sec:Cancellation}
With version 4.0, the OpenMP language introduced the concept of cancellation to allow for ending parallel execution of OpenMP-parallel regions.
This becomes handy in algorithms that traverse some sort of search space or data structure in parallel, and that should stop as soon as a condition is met or if the search found a hit.

OpenMP defines two directives for programmers to use in the context of cancellation.
First, the \code{cancellation} construct causes the encountering thread to request stopping execution for a subset of the inner-most \code{parallel}, \code{sections}, \code{for} or \code{do} region or a set of tasks that is defined by the \code{taskgroup} construct (see Section~\ref{sec:Taskgroup}).
The syntax of the \code{cancellation} construct define this subset through clauses that can be given to the construct.

Once a thread encounters a \code{cancellation} construct, it requests cancellation from the OpenMP implementation and then proceeds to the end of the canceled region.
For instance, if cancellation for an OpenMP \code{taskgroup} is performed, the encountering thread proceeds to the end of the current task.
Threads that reach a \code{cancellation point} construct, test if cancellation has been activated and then also proceed to the end of the canceled region.
For the example of OpenMP tasks, this means that these threads stop executing the current task and proceed to the end of the task region.

\begin{figure}
\begin{minted}{c}
binary_tree_t*
search_tree(binary_tree_t* tree,
            int value) {
  binary_tree_t* found = NULL;
  if (tree) {
    if (tree->value == value) {
      found = tree;
    }
    else {
#pragma omp task shared(found)
      {
        binary_tree_t* found_left;
        found_left =
          search_tree(tree->left,
                      value);
        if (found_left) {
#pragma omp atomic write
          found = found_left;
#pragma omp cancel taskgroup
        }
      }
#pragma omp task shared(found)
      {
        binary_tree_t* found_right;
        found_right =
          search_tree(tree->right,
                      value);
        if (found_right) {
#pragma omp atomic write
          found = found_right;
#pragma omp cancel taskgroup
        }
      }
#pragma omp taskwait
    }
  }
  return found;
}
\end{minted}
\caption{Example of cancellation while traversing a binary search tree.\label{fig:Cancellation}}
\end{figure}

Figure~\ref{fig:Cancellation} shows an example how to apply cancellation to end a search in a binary tree if the search value has been found.
Ignoring the OpenMP directives, the traversal code is quite simple.
If the value of the current tree node matches the search value, the recursion aborts and returns the tree node containing the value.
Otherwise, the code first traverses the left sub-tree and then the right sub-tree.

With OpenMP tasking added, the code traverses the left and right sub-tree in parallel.
Once one of the generated tasks hits the search value, it stores a pointer to the the corresponding tree node and requests cancellation.
As queued tasks will be discarded, no new work will be generated and parallel computation will stop.
Please note, this implementation of the parallel tree search is not guaranteed to find the first tree node that contains the search value.

