\subsection{Descriptive or Prescriptive Semantics}
\label{sub:descriptive}

A debate has emerged within the high performance language
community about whether programming languages should 
provide \emph{descriptive} or \emph{prescriptive} semantics. 
A language provides descriptive semantics if its constructs 
describe the computation that should be performed but provides 
the compiler and runtime the flexibility to determine exactly 
how the computation is performed. Alternatively, prescriptive
semantics not only describe the required computation but 
also all details of how to perform it.

Our position is that the debate is misguided for two reasons.
First, the debate assumes a binary choice between the two 
types of semantics. However, almost all languages have constructs 
that are descriptive while others are (more) prescriptive. 
Specifically within the HPC community, the claim has been 
made that OpenACC is descriptive while OpenMP is
prescriptive~\cite{juckeland2016isc,wolfe16descriptive}. While OpenACC provides
more descriptive constructs in its most recent version than OpenMP does, the
\texttt{acc parallel loop} directive is prescriptive since sometimes users want
to \emph{know} that a given loop will be parallelized. 

Alternatively, most OpenMP defaults allow the compiler freedom
to choose details about how the computation is performed. Even
the \texttt{num\_threads} clause of the \texttt{parallel} construct,
which many believe to be among its most prescriptive mechanisms,
allows the compiler and runtime to determine if the number of
threads requested are available. If that many threads are not
available, the compiler and runtime have the flexibility to
determine how many threads to use. So, one may see the issue 
as where to place a language, or even its constructs, on a 
continuum of possible semantics.

Second, even the view of placement on a continuum is overly
limited and fails to address the overall preference of 
programmers. Specifically, they would prefer that the compiler
and runtime would always ``do the right thing'' given a 
description of the computation to perform. However, in reality,
compilers and runtimes often do not. In these instances,
programmers prefer to have the ability to override the
decisions and to specify exactly how to perform the computation.

For these reasons, the emerging OpenMP philosophy is to provide
mechanisms that describe the computation to perform and that
prescribe as much or as little as the programmer desires about
how to perform it. As a first step, OpenMP 5.0 will add the 
\texttt{concurrent} construct, which only informs the compiler 
and runtime that a loop nest is easily parallelized. In the 
longer run, we are exploring mechanisms that specify that the 
intent of a clause or a construct is fully descriptive or prescriptive.
